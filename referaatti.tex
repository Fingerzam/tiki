% --- Template for thesis / report with tktltiki2 class ---

\documentclass[finnish]{tktltiki2}

% tktltiki2 automatically loads babel, so you can simply
% give the language parameter (e.g. finnish, swedish, english, british) as
% a parameter for the class: \documentclass[finnish]{tktltiki2}.
% The information on title and abstract is generated automatically depending on
% the language, see below if you need to change any of these manually.
% 
% Class options:
% - grading                 -- Print labels for grading information on the front page.
% - disablelastpagecounter  -- Disables the automatic generation of page number information
%                              in the abstract. See also \numberofpagesinformation{} command below.
%
% The class also respects the following options of article class:
%   10pt, 11pt, 12pt, final, draft, oneside, twoside,
%   openright, openany, onecolumn, twocolumn, leqno, fleqn
%
% The default font size is 11pt. The paper size used is A4, other sizes are not supported.
%
% rubber: module pdftex

% --- General packages ---

\usepackage[utf8]{inputenc}
\usepackage{lmodern}
\usepackage{microtype}
\usepackage{amsfonts,amsmath,amssymb,amsthm,booktabs,color,enumitem,graphicx}
\usepackage[pdftex,hidelinks]{hyperref}

% Automatically set the PDF metadata fields
\makeatletter
\AtBeginDocument{\hypersetup{pdftitle = {\@title}, pdfauthor = {\@author}}}
\makeatother

% --- Language-related settings ---
%
% these should be modified according to your language

% babelbib for non-english bibliography using bibtex
\usepackage[fixlanguage]{babelbib}
\selectbiblanguage{finnish}

% add bibliography to the table of contents
\usepackage[nottoc,numbib]{tocbibind}
% tocbibind renames the bibliography, use the following to change it back
\settocbibname{Lähteet}

% --- Theorem environment definitions ---

\newtheorem{lau}{Lause}
\newtheorem{lem}[lau]{Lemma}
\newtheorem{kor}[lau]{Korollaari}

\theoremstyle{definition}
\newtheorem{maar}[lau]{Määritelmä}
\newtheorem{ong}{Ongelma}
\newtheorem{alg}[lau]{Algoritmi}
\newtheorem{esim}[lau]{Esimerkki}

\theoremstyle{remark}
\newtheorem*{huom}{Huomautus}

\newcommand{\set}[1]{\left\{ #1 \right\}}
\newcommand{\nat}{\mathbb{N}}


% --- tktltiki2 options ---
%
% The following commands define the information used to generate title and
% abstract pages. The following entries should be always specified:

\title{Paikallisuus hajautetuissa verkkoalgoritmeissa}
\author{Juhana Laurinharju}
\date{\today}
\level{Tieteellinen kirjoittaminen}
\abstract{Tiivistelmä.}

% The following can be used to specify keywords and classification of the paper:

%\keywords{avainsana 1, avainsana 2, avainsana 3}
%\classification{} % classification according to ACM Computing Classification System (http://www.acm.org/about/class/)
                  % This is probably mostly relevant for computer scientists

% If the automatic page number counting is not working as desired in your case,
% uncomment the following to manually set the number of pages displayed in the abstract page:
%
% \numberofpagesinformation{16 sivua + 10 sivua liitteissä}
%
% If you are not a computer scientist, you will want to uncomment the following by hand and specify
% your department, faculty and subject by hand:
%
% \faculty{Matemaattis-luonnontieteellinen}
% \department{Tietojenkäsittelytieteen laitos}
% \subject{Tietojenkäsittelytiede}
%
% If you are not from the University of Helsinki, then you will most likely want to set these also:
%
% \university{Helsingin Yliopisto}
% \universitylong{HELSINGIN YLIOPISTO --- HELSINGFORS UNIVERSITET --- UNIVERSITY OF HELSINKI} % displayed on the top of the abstract page
% \city{Helsinki}
%


\begin{document}

% --- Front matter ---

\maketitle        % title page
\makeabstract     % abstract page

\tableofcontents  % table of contents
\newpage          % clear page after the table of contents


% --- Main matter ---

\section{Johdanto}

Artikkelissa \emph{Locality in Distributed Graph Algorithms}~\cite{linial92}
Linial näyttää, ettei hajautettu algoritmi voi 3-värittää $n$ solmun sykliä
alle ${\log^* n}$ kierroksessa. Tämä alaraja on tiukka, sillä Colen ja
Vishkinin~\cite{colevishkin86} algoritmilla $n$-syklin voi 3-värittää ${\log^*
n}$ kierroksessa.

\section{Laskennan malli}

Olkoon $G = (V,E)$ suuntaamaton verkko. Verkon jokaisessa solmussa $v \in V$ on
tietokone. Laskenta koostuu kommunikointikierroksista. Yhden
kommunikaatiokierroksen aikana jokainen solmu voi:

\begin{enumerate}
    \item suorittaa mielivaltaista laskentaa
    \item lähettää viestin jokaiselle naapurilleen
    \item vastaanottaa naapureiden lähettämät viestit
\end{enumerate}

\section{Verkon väritys}

Verkko on väritetty, jos jokaiseen solmuun $v \in V$ on liitetty jokin väri ja
kahdella vierekkäisellä solmulla ei koskaan ole samaa väriä. Tarkemmin, verkon
$G = (V,E)$ \emph{solmuväritys} on kuvaus $c : V \rightarrow \set{1,\dots,k}$
jollain luonnollisella luvulla $k \in \nat$. Lisäksi vaaditaan, että jos
verkossa on kaari solmusta $v$ solmuun $u$, eli $vu \in E$, niin $c(v) \neq
c(u)$.

Verkon voi värittää $k$:lla värillä jos löytyy yllä olevan ehdon täyttävä
kuvaus $c : V \rightarrow \set{1,\dots,k}$. Tällaista väritystä kutsutaan
$k$-väritykseksi.

TODO: tähän pari kuvaa väritetyistä verkoista

\section{Sykli}

Verkko on sykli, jos se on yhtenäinen ja sen jokaisella solmulla on tasan kaksi
naapuria.

TODO: kuva syklistä

Syklin voi aina värittää kolmella värillä.

TODO: kuva 3-väritetystä syklistä

\section{Iteroitu logaritmi $\log^*$}

Iteroitu logaritmi $\log^*$ kertoo kuinka monta kertaa luvusta täytyy ottaa
logaritmi, kunnes lopputulos on korkeintaan yksi.

\begin{equation*}
    \log^* x =
      \begin{cases}
          0               &\text{, jos } x \leq 1 \\
          1 + \log^* (\log x) &\text{, muutoin}
      \end{cases}
\end{equation*}

Esimerkiksi
%
\begin{align*}
    \log^* 16 &= \log^* 2^{2^2} = 1 + \log^* 2^2 \\
              &= 2+ \log^* 2 = 3 + \log^* 1 = 3 \\
\intertext{ja}
    \log^* 65536 &= \log^* 2^{2^{2^2}} = 1 + \log^* 16 \\
                 &= 4,
\end{align*}
%
mistä voi huomata, että $\log^* n$ on arvoltaan pienempi kuin 5, kun $n <
2^{65536}$. Iteroitu logaritmi on siis äärimmäisen hitaasti kasvava funktio.

\section{todo}

\begin{itemize}
    \item $\Omega(\log^* n)$ alaraja syklissä
    \item löytyy $O(\log^* n)$ algoritmi
    \item viitteet verkkoteoreettisten termien määritelmiin?
    \item viite laskennan mallin määritelmän yhteyteen?
\end{itemize}


% --- Back matter ---
%
% bibtex is used to generate the bibliography. The babplain style
% will generate numeric references (e.g. [1]) appropriate for theoretical
% computer science. If you need alphanumeric references (e.g [Tur90]), use
%
% \bibliographystyle{babalpha}
%
% instead.

\bibliographystyle{babplain}
\bibliography{references-fi}


\end{document}
